%%%%%%%%%%%%%%%%%%%%%%%%%%%%%%%%%%%%%%%%%%%%%%%%%%%%%%%%%%%%%%%%%%
%%%%%%%% ICML 2017 EXAMPLE LATEX SUBMISSION FILE %%%%%%%%%%%%%%%%%
%%%%%%%%%%%%%%%%%%%%%%%%%%%%%%%%%%%%%%%%%%%%%%%%%%%%%%%%%%%%%%%%%%

% Use the following line _only_ if you're still using LaTeX 2.09.
%\documentstyle[icml2017,epsf,natbib]{article}
% If you rely on Latex2e packages, like most moden people use this:
\documentclass{article}

% use Times
\usepackage{times}
% For figures
\usepackage{graphicx} % more modern
%\usepackage{epsfig} % less modern
\usepackage{subfigure} 

% For citations
\usepackage{natbib}

% For algorithms
\usepackage{algorithm}
\usepackage{algorithmic}

% As of 2011, we use the hyperref package to produce hyperlinks in the
% resulting PDF.  If this breaks your system, please commend out the
% following usepackage line and replace \usepackage{icml2017} with
% \usepackage[nohyperref]{icml2017} above.
\usepackage{hyperref}

% Packages hyperref and algorithmic misbehave sometimes.  We can fix
% this with the following command.
\newcommand{\theHalgorithm}{\arabic{algorithm}}

% Employ the following version of the ``usepackage'' statement for
% submitting the draft version of the paper for review.  This will set
% the note in the first column to ``Under review.  Do not distribute.''
\usepackage{icml2017} 

% Employ this version of the ``usepackage'' statement after the paper has
% been accepted, when creating the final version.  This will set the
% note in the first column to ``Proceedings of the...''
%\usepackage[accepted]{icml2017}


% The \icmltitle you define below is probably too long as a header.
% Therefore, a short form for the running title is supplied here:
\icmltitlerunning{Submission and Formatting Instructions for ICML 2017}

\begin{document} 

\twocolumn[
\icmltitle{Counterfactual Fairness}

% It is OKAY to include author information, even for blind
% submissions: the style file will automatically remove it for you
% unless you've provided the [accepted] option to the icml2017
% package.

% list of affiliations. the first argument should be a (short)
% identifier you will use later to specify author affiliations
% Academic affiliations should list Department, University, City, Region, Country
% Industry affiliations should list Company, City, Region, Country

% you can specify symbols, otherwise they are numbered in order
% ideally, you should not use this facility. affiliations will be numbered
% in order of appearance and this is the preferred way.
\icmlsetsymbol{equal}{*}

\begin{icmlauthorlist}
\icmlauthor{Cieua Vvvvv}{goo}
\icmlauthor{Iaesut Saoeu}{ed}

\end{icmlauthorlist}

\icmlaffiliation{goo}{Googol ShallowMind, New London, Michigan, USA}
\icmlaffiliation{ed}{University of Edenborrow, Edenborrow, United Kingdom}

\icmlcorrespondingauthor{Cieua Vvvvv}{c.vvvvv@googol.com}

% You may provide any keywords that you 
% find helpful for describing your paper; these are used to populate 
% the "keywords" metadata in the PDF but will not be shown in the document
\icmlkeywords{boring formatting information, machine learning, ICML}

\vskip 0.3in
]

% this must go after the closing bracket ] following \twocolumn[ ...

% This command actually creates the footnote in the first column
% listing the affiliations and the copyright notice.
% The command takes one argument, which is text to display at the start of the footnote.
% The \icmlEqualContribution command is standard text for equal contribution.
% Remove it (just {}) if you do not need this facility.

\printAffiliationsAndNotice{}  % leave blank if no need to mention equal contribution
%\printAffiliationsAndNotice{\icmlEqualContribution} % otherwise use the standard text.
%\footnotetext{hi}

\begin{abstract} 
TODO
\end{abstract} 

\section{Introduction}
\label{introduction}
%!TEX root=ricardo_draft.tex
% ml is now everywhere
Machine learning is now used in fields as diverse as credit scoring (CITE), crime prediction (CITE), and loan assessment (CITE). As machine learning enters these new areas it is necessary for the modeler to think beyond the simple objective of maximizing prediction accuracy.

% in these new ml fields, we cannot discriminate
% discrimination can happen in multiple ways
% - direct discrimination
In particular, for many of these new applications, it is crucial to consider whether the predictions of a machine learning model are \emph{fair}. For instance, imagine a bank wishes to train a machine learning model to predict whether or not an individual should be given a loan to buy a house. The bank wishes to use historical lending data on whether or not a loan was paid back, along with personal information on individuals. If the bank simply tries to learn a model that accurately predicts who to loan to solely based on whether the loan will be paid back, it may unjustly favor giving loans to applicants of a particular demographic group, due to historical and present prejudices. The Obama Administration released a report that precisely describes this, and urged machine learning practicioners to analyze ``how technologies can deliberately or inadvertently perpetuate, exacerbate, or mask discrimination"\footnote{https://obamawhitehouse.archives.gov/blog/2016/05/04/big-risks-big-opportunities-intersection-big-data-and-civil-rights}.

As a result, there has been immense interest recently in designing machine learning algorithms that make fair predictions. (CITE A MILLION PAPERS). In large part each work focuses on formalizing fairness into a concrete definition that can be tested, and for which algorithms can be developed. By defining fairness concretely, one can design specific algorithms to satisfy such definitions. 



%  This could be due to many factors that are observed and unobserved in the features of a given dataset such as:
% \begin{itemize}
% \item (\emph{observed}) race could be a feature in the dataset, thus any algorithm using this feature for prediction is directly using race to discriminate.
% \item (\emph{observed}) other features could be proxies for race, such as where an individual currently lives.
% \item (\emph{unobserved}) there may exist historical biases that make it more difficult for certain races to secure employment, thereby making a direct comparison between two individuals of different races unfair and could actually harm prediction (as a high-earning individual in one race may have had to work much harder to obtain work than a similar individual in another race, thus it may be crucial to look beyond observed features and favor the hard-working individual).
% \item (\emph{unobserved}) the historical bank data may favor giving loans to individuals of a certain race because of prejudiced lenders.
% \end{itemize}
% These are just a few possible sources of unfairness that an unaware classifier could exploit in order to make more accurate predictions.

% there's a lot of interest in this
% There has been immense interest recently in designing machine learning algorithms that make fair predictions. (CITE A MILLION PAPERS). In large part each work focuses on formalizing fairness into a concrete definition that can be tested, and for which algorithms can be developed. 

% By defining fairness concretely, one can design specific algorithms to satisfy such definitions. The hope is that such algorithms can begin to address the calls of various governing bodies about the need for fairness in automated algorithms. For instance, in the United States the Obama Administration has issued two reports (CITE), the first warning individuals about ``the potential of encoding discrimination in automated decisions" and the second describing ``how technologies can deliberately or inadvertently perpetuate, exacerbate, or mask discrimination"\footnote{https://obamawhitehouse.archives.gov/blog/2016/05/04/big-risks-big-opportunities-intersection-big-data-and-civil-rights}. Thus there is significant interest defining fairness in order to address it.

% a lot of this work just proposes a new definition of fairness and checks it
% these definitions may or may not be appropriate for a given problem
In large part, the initial work on fairness in machine learning has focused on formalizing the above definitions and using them to solve a discrimination problem in a certain dataset. Unfortunately, for a practitioner, law-maker, judge, or anyone else who is interested in implementing algorithms that control for discrimination, it can be difficult to decide which definition of fairness to choose for the task at hand. Indeed, we demonstrate that depending on the relationship between a sensitive attribute and the data certain definitions of fairness can actually \emph{increase discrimination}.

% we propose a way to model data that allows a practitioner to assess what definitions of fairness are right for the problem at hand, and algorithms to ensure fairness
% OR
% we propose a way to interpret fairness...
% a) relationship between fairness and causality
% b) use pearl's models
% c) having an explicit model allows us to test fairness with the assumptions laid bare
% tension: Pearl already talks about discrimination, so we aren't really inventing new models. Are we even new in using these models to talk about fairness? Maybe... Pearl talks about variables that we might want to compute counterfactuals for in order to see if discrimination is happening.  
% Our proposal is:
% - situate a sensitive variable in a graph (not new).
% - Look at old definitions and see if anything bad could happen (new). 
% - Then define counterfactual fairness (new). 
% - Modeling helps us see where the weaknesses are in our assumptions and definitions (maybe not new)
% We don't want to see if every definition is counterfactually fair because then we're like everyone else, saying our definition is best
% 

In this work, we describe how techniques from causal inference can be used to formalize questions of fair prediction. Specifically, we develop a technique to leverage the causal models of Pearl \cite{pearl2009causal} to model the relationship between the sensitive attribute and data. Our contributions are as follows:
\begin{enumerate}
    \item We model questions of fairness within a causal framework. This allows us to directly model \emph{how} unfairness affects the data at hand.
    \item We introduce \emph{counterfactual fairness}, which enforces that a distribution over possible predictions for an individual should remain unchanged, in a world where an individual's sensitive attribute was changed.
    \item We analyze how enforcing existing definitions of fairness for different data may or may not lead to fair predictions.
    \item We devise techniques learning predictors that are counterfactually fair.
\end{enumerate}
%We demonstrate that by explicitly representing fairness within a causal model it becomes easy to critique different definitions of fairness as well the prediction methods that aim to accomplish these notions of fairness.











% RETHINK SPIN, ALWAYS RETHINK

TODO

\section{Background}
\label{background}

\subsection{Fairness}

TODO

\subsection{Causal Models and Counterfactuals}

We will follow the framework of \cite{pearl:00}, where a causal
model is a triple $(U, V, F)$ of sets such that
\begin{itemize}
\item $U$ is a set of {\bf background} variables\footnote{These are
  sometimes called {\bf exogeneous variables}, but the fact that members of $U$
  might depend on each other is not relevant to what follows.}, which are generated by factors
outside of our potential control;
\item $V$ is a set of {\bf endogenous} variables, where each member is determined by
  other variables in $U \cup V$;
\item $F$ is a set of functions $\{f_1, \dots, f_n\}$, one for each $V_i \in V$, such
that $V_i = f_i(pa_i, U_{pa_i})$, $pa_i \subseteq V \backslash
\{V_i\}$ and $U_{pa_i} \subseteq U$. Such equations are also known as
{\bf structural equations} \citep{bol:89}.
\end{itemize}

The notation ``$pa_i$'' is motivated by the extra assumption that the
model factorizes according to a directed acyclic graph (DAG). That is,
define a directed graph $\mathcal G$ where each node corresponds to an
element of $U \cup V$, and each edge $V_i \leftarrow X$ is added if
and only if $X \in pa_i \cup U_{pa_i}$. We assume $\mathcal G$ is
acyclic.

The model is causal in the sense that, for a given probability model
$p(U)$ for the background variables, it entails the distribution of a
subset of $V$ given an {\bf intervention} in another subset of $V$.
The operational meaning of an intervention on $V_i$ at value $v$ is
the substitution of the equation $V_i = f_i(pa_i, U_{pa_i})$ with the
equation $V_i = v$. This captures the idea of an agent modifying a
system while being external to it. For instance, this can happen as a
randomized controlled trial that overrides the value of $V_i$ with a
treatment that sets it at $v$, a value chosen at random independently
of any other causes of the system. Pearl's do-calculus
\citep{pearl:00} provides a way of identifying features of 
interventional distributions, when possible, using only (estimates of) the joint
distribution of $V$ and the causal DAG.

Compared to independence constraints given by a DAG, the full
specification of $F$ requires much stronger assumptions but also leads
to much more specific claims. In particular, it allows for the
calculation of {\bf counterfactual} quantities. Without going into a
detailed coverage of the topic, consider the following counterfactual
statement, ``the value of $Y$ had $X$ been $x$'', for two endogenous
variables $X$ and $Y$ in a causal model. By assumption, the state of
any endogenous variable is fully determined by
the background variables and structural equations. The counterfactual is
modeled as the solution for $Y$ for a given $U = u$ where the equation(s)
for $X$ is (are) replaced with $X = x$.  We denote it by $Y_{X \leftarrow x}(u)$
\cite{pearl:00}.

Counterfactual inference, as specified by a causal model $(U, V, F)$,
is the computation of probabilities $P(Y_{X \leftarrow x}(U)\ |\ W =
w)$, where $W$, $X$ and $Y$ are subsets of $V$. Inference proceeds in
three steps, as explained in more detail in Chapter 4 of
\cite{pearl:16}:
\begin{enumerate}
\item For a given prior on $U$, compute the posterior distribution of $U$ given the evidence $W = w$;
\item Substitute the equations for $X$ with the interventional values $x$, resulting
     in the modified set of equations $F_x$;
\item Compute the implied distribution on the remaining elements of $V$
     using $F_x$ and the posterior $P(U\ | W = w)$.
\end{enumerate}

\section{Counterfactual Fairness}
\label{sec:count_fair}

\subsection{Definition}

TODO. Here the main definition is introduced, and how it relates to ``path deletion'',
including the core example of $A \rightarrow X \rightarrow Y$, with two latent
variables $U_x \rightarrow X$ and $U_y \rightarrow Y$, arguing that one might judge
that the path from $A$ to $Y$ via is due to an unfair mechanism and that we need a
notion of ``closest world''.

\subsection{Other Examples}

TODO. Here the examples and their motivation can be as follows:

\begin{itemize}
\item something analogous to the red car example: $A$ is not a cause of
  $Y$ but might indirectly bias the result even without using $A$ as a predictor;
\item something with selection bias, maybe a toy version of COMPAS;
\item something where an unfair judgement (say, credit score) that can be potentially
  considered as a target variable, and an
  ``objective target'' (say, defaulting on a loan) are present, and
  what the recommendation is
\end{itemize}

\section{Methods and Assessment}
\label{sec:methods}

TODO. General recipe on how to put a model together, including the practical
issues of $U$ and $A$ being dependent due to randomness of $U$.

\subsection{Special Cases}

Talk about linear models, including those based on PCA/ICA.

\subsection{Criticisms}

TODO. Criticism on difficulty of untestability of assumptions.
More on assumptions used by other methods, including causal
estimation methods based on additive error models.

\section{Experiments}

TODO

\section{Conclusion}

TODO

\bibliography{rbas}
\bibliographystyle{icml2017}

\end{document} 


% This document was modified from the file originally made available by
% Pat Langley and Andrea Danyluk for ICML-2K. This version was
% created by Lise Getoor and Tobias Scheffer, it was slightly modified  
% from the 2010 version by Thorsten Joachims & Johannes Fuernkranz, 
% slightly modified from the 2009 version by Kiri Wagstaff and 
% Sam Roweis's 2008 version, which is slightly modified from 
% Prasad Tadepalli's 2007 version which is a lightly 
% changed version of the previous year's version by Andrew Moore, 
% which was in turn edited from those of Kristian Kersting and 
% Codrina Lauth. Alex Smola contributed to the algorithmic style files.  
