
\subsection{Fairness}

TODO

\subsection{Causal Models and Counterfactuals}

We will follow the framework of \cite{pearl:00}, where a causal
model is a triple $(U, V, F)$ of sets such that
\begin{itemize}
\item $U$ is a set of {\bf background} variables\footnote{These are
  sometimes called {\bf exogeneous variables}, but the fact that members of $U$
  might depend on each other is not relevant to what follows.}, which are generated by factors
outside of our potential control;
\item $V$ is a set of {\bf endogenous} variables, where each member is determined by
  other variables in $U \cup V$;
\item $F$ is a set of functions $\{f_1, \dots, f_n\}$, one for each $V_i \in V$, such
that $V_i = f_i(pa_i, U_{pa_i})$, $pa_i \subseteq V \backslash
\{V_i\}$ and $U_{pa_i} \subseteq U$. Such equations are also known as
{\bf structural equations} \citep{bol:89}.
\end{itemize}

The notation ``$pa_i$'' is motivated by the extra assumption that the
model factorizes according to a directed acyclic graph (DAG). That is,
define a directed graph $\mathcal G$ where each node corresponds to an
element of $U \cup V$, and each edge $V_i \leftarrow X$ is added if
and only if $X \in pa_i \cup U_{pa_i}$. We assume $\mathcal G$ is
acyclic.

The model is causal in the sense that, for a given probability model
$p(U)$ for the background variables, it entails the distribution of a
subset of $V$ given an {\bf intervention} in another subset of $V$.
The operational meaning of an intervention on $V_i$ at value $v$ is
the substitution of the equation $V_i = f_i(pa_i, U_{pa_i})$ with the
equation $V_i = v$. This captures the idea of an agent modifying a
system while being external to it. For instance, this can happen as a
randomized controlled trial that overrides the value of $V_i$ with a
treatment that sets it at $v$, a value chosen at random independently
of any other causes of the system. Pearl's do-calculus
\citep{pearl:00} provides a way of identifying features of 
interventional distributions, when possible, using only (estimates of) the joint
distribution of $V$ and the causal DAG.

Compared to independence constraints given by a DAG, the full
specification of $F$ requires much stronger assumptions but also leads
to much more specific claims. In particular, it allows for the
calculation of {\bf counterfactual} quantities. Without going into a
detailed coverage of the topic, consider the following counterfactual
statement, ``the value of $Y$ had $X$ been $x$'', for two endogenous
variables $X$ and $Y$ in a causal model. By assumption, the state of
any endogenous variable is fully determined by
the background variables and structural equations. The counterfactual is
modeled as the solution for $Y$ for a given $U = u$ where the equation(s)
for $X$ is (are) replaced with $X = x$.  We denote it by $Y_{X \leftarrow x}(u)$
\cite{pearl:00}.

Counterfactual inference, as specified by a causal model $(U, V, F)$,
is the computation of probabilities $P(Y_{X \leftarrow x}(U)\ |\ W =
w)$, where $W$, $X$ and $Y$ are subsets of $V$. Inference proceeds in
three steps, as explained in more detail in Chapter 4 of
\cite{pearl:16}:
\begin{enumerate}
\item For a given prior on $U$, compute the posterior distribution of $U$ given the evidence $W = w$;
\item Substitute the equations for $X$ with the interventional values $x$, resulting
     in the modified set of equations $F_x$;
\item Compute the implied distribution on the remaining elements of $V$
     using $F_x$ and the posterior $P(U\ | W = w)$.
\end{enumerate}