%!TEX root=ricardo_draft.tex
\subsection{Definition}

TODO. Here the main definition is introduced, and how it relates to ``path deletion'',
including the core example of $A \rightarrow X \rightarrow Y$, with two latent
variables $U_x \rightarrow X$ and $U_y \rightarrow Y$, arguing that one might judge
that the path from $A$ to $Y$ via is due to an unfair mechanism and that we need a
notion of ``closest world''.

\subsection{Other Examples}
To get an intution for what counterfactual fairness means, we will begin by describing a few possible real-world scenarios. For each of these we will describe what counterfactually-fair and counterfactually-unfair predictors look like.

\paragraph{Scenario 1: The Red Car.}
Imagine a car insurance company wants a quick, anonymous way to determine how to price insurance for different car owners by predicting their accident likelihood $Y$. They've noticed that there is a correlation between driving a red car $X$ and a higher rate of automobile accidents. Thus they would like to increase the insurance for all red car drivers. 

Imagine what's really going on is shown in Figure~\ref{figure.simple_models} (\emph{Left}). The correlation between have a red car $X$ and accidents $Y$ is due to an `aggressiveness' factor $U$: aggressiveness causes individuals to be in accidents more often, and it also attracts them to red cars. Unfortunately, the red car feature $X$ is also affected by an individual's race $A$. 

Thus, simply using $X$ to predict $Y$ would seem to be an unfair prediction because it may charge individual's of a certain race more than others. Counterfactual fairness agrees with this. 

\begin{lem}
The predictor $\hat{Y}(X)$ is not counterfatually fair for the model in Figure~\ref{figure.simple_models} (\emph{Left}).
\end{lem}

\begin{proof}
For simplicity, let's imagine that the equations described by the model in Figure~\ref{figure.simple_models} (\emph{Left}) are deterministic and linear:
\begin{align}
X = \alpha A + \beta U, \;\;\;\; Y = \gamma U \nonumber
\end{align}
where the coefficients $\alpha,\beta,\gamma$ are given (in practice we will estimate these). We can test whether a predictor $\hat{Y}(X)$ is counterfactually-fair using the procedure described in Section~\ref{subsec:cmc}:
\begin{itemize}
\item Compute $U$ given observations of $X,Y,A$, by solving for $U$.
\item Substitute the equations involving $A$ with an interventional value $a'$ (i.e., this says: `what happens if the race of an individual were changed')
\item Compute the variables $X,Y$ with the interventional value $a'$
\end{itemize}
In this deterministic case we achieve counterfactual fairness only if the predicted value $\hat{Y}(X)$ is identical before and after we change $A$. However, note that changing $A$ causes $X$ to change. Thus $\hat{Y}(X)$ must be different and so our model based on using $X$ to predict $Y$ is not counterfactually fair.
\end{proof}
Note that the above proof is without loss of generality: any model constructed using $X$ alone will not be counterfactually fair as changing $A$ will always have an influence on $X$.


TODO. Here the examples and their motivation can be as follows:

\begin{itemize}
\item something analogous to the red car example: $A$ is not a cause of
  $Y$ but might indirectly bias the result even without using $A$ as a predictor;
\item something with selection bias, maybe a toy version of COMPAS;
\item something where an unfair judgement (say, credit score) that can be potentially
  considered as a target variable, and an
  ``objective target'' (say, defaulting on a loan) are present, and
  what the recommendation is
\end{itemize}