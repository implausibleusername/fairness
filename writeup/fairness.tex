\subsection{Definition}

TODO. Here the main definition is introduced, and how it relates to ``path deletion'',
including the core example of $A \rightarrow X \rightarrow Y$, with two latent
variables $U_x \rightarrow X$ and $U_y \rightarrow Y$, arguing that one might judge
that the path from $A$ to $Y$ via is due to an unfair mechanism and that we need a
notion of ``closest world''.

\subsection{Other Examples}
To get an intution for what counterfactual fairness means, we will begin by describing a few possible real-world scenarios. For each of these we will describe what counterfactually-fair and counterfactually-unfair predictors look like.

\paragraph{Scenario 1: The Red Car.}
Imagine a car insurance company wants a quick, anonymous way to determine how to price insurance for different car owners by predicting their accident likelihood $Y$. They've noticed that there is a correlation between driving a red car $X$ and a higher rate of automobile accidents. Thus they would like to increase the insurance for all red car drivers. 

Imagine what's really going on is shown in Figure 1 (\emph{Left}). The correlation between have a red car $X$ and accidents $Y$ is due to an `aggressiveness' factor $U$: aggressiveness causes individuals to be in accidents more often, and it also attracts them to red cars. Unfortunately, the red car feature $X$ is also affected by an individual's race $A$. Thus, simply using $X$ to predict $Y$ would seem to be an unfair prediction because it may charge individual's of a certain race more than others. Counterfactual fairness agrees with this. We can test whether a predictor $\hat{Y}(X)$ is counterfactually-fair using the procedure described in 



TODO. Here the examples and their motivation can be as follows:

\begin{itemize}
\item something analogous to the red car example: $A$ is not a cause of
  $Y$ but might indirectly bias the result even without using $A$ as a predictor;
\item something with selection bias, maybe a toy version of COMPAS;
\item something where an unfair judgement (say, credit score) that can be potentially
  considered as a target variable, and an
  ``objective target'' (say, defaulting on a loan) are present, and
  what the recommendation is
\end{itemize}